\chapter{绪论}%大标题




\zhlipsum[1]

\section{术语测试}

\myterm{你好}{nihao}
\myterm[1]{嗨}{hi}
\myterm{你好}{nihao}
\myterm[1]{嗨}{hi}
\myterm{你好}{nihao}
\myterm[1]{嗨}{hi}
\myterm{你好}{nihao}
\myterm[1]{嗨}{hi}
\myterm{你好}{nihao}
\myterm[1]{嗨}{hi}
\myterm{你好}{nihao}
\myterm[1]{嗨}{hi}
\myterm{你好}{nihao}
\myterm[1]{嗨}{hi}
\myterm{你好}{nihao}
\myterm[1]{嗨}{hi}
\myterm{你好}{nihao}
\myterm[1]{嗨}{hi}


\section{表格测试}
\zhlipsum[1]

\begin{table}[h!]
	\centering 
	\caption{三线表示例}
	\begin{tblr}{lccr}
		\toprule
		Alpha   & Beta  & Gamma  & Delta \\
		\midrule
		Epsilon & Zeta  & Eta    & Theta \\
		Iota    & Kappa & Lambda & Mu    \\
			OK    & Good & Nice & Well    \\
		\bottomrule
	\end{tblr}
\end{table}







\section{参考文献测试}

可以参看 \parencite{Weng:2017jyh,Biro:2017flp,Hod:2017ssh}。
张三 \cite{叶普1993关于对瞬心的动量矩定理}提出。
张三 \parencite{爱因斯坦文集2009}提出。
张三 \cite{叶普1993关于对瞬心的动量矩定理}提出。
张三 \cite{叶普1993关于对瞬心的动量矩定理,2003张量分析,lanczos1986variational}提出。
海上生明月\footnote{脚注测试1},
海上生明月\footnote{脚注测试2,脚注测试2}。
文献 \cite{2003张量分析}中说明。
文献 \cite{叶普1993关于对瞬心的动量矩定理,2003张量分析,lanczos1986variational}中 说明。
张三 \parencite{爱因斯坦文集2009,叶普1993关于对瞬心的动量矩定理,2003张量分析,lanczos1986variational}提出。

可以参看\cite{Litvinova:2017jnd,Li:2017kdj}。



\section{脚注测试}%二级标题

\zhlipsum[2-3]
海上生明月\footnote{脚注测试1}
海上生明月\footnote{脚注测试2,脚注测试2}
海上生明月\footnote{脚注测试1}
海上生明月\footnote{脚注测试2,脚注测试2}
海上生明月\footnote{脚注测试1}
海上生明月\footnote{脚注测试2,脚注测试2}
海上生明月\footnote{脚注测试1}
海上生明月\footnote{脚注测试2,脚注测试2}
海上生明月\footnote{脚注测试1}
海上生明月\footnote{脚注测试2,脚注测试2}
海上生明月\footnote{脚注测试1}
海上生明月\footnote{脚注测试2,脚注测试2}
海上生明月\footnote{脚注测试1}
海上生明月\footnote{脚注测试2,脚注测试2}
海上生明月\footnote{脚注测试1}
海上生明月\footnote{脚注测试2,脚注测试2}
海上生明月\footnote{脚注测试1}
海上生明月\footnote{脚注测试2,脚注测试2}
海上生明月\footnote{脚注测试1}
海上生明月\footnote{脚注测试2,脚注测试2}
海上生明月\footnote{脚注测试1}
海上生明月\footnote{脚注测试2,脚注测试2}
海上生明月\footnote{脚注测试1}
海上生明月\footnote{脚注测试2,脚注测试2}

\subsection{理论预言}%三级标题



\zhlipsum[3]


\begin{figure}[h!]
	\centering
	\begin{subfigure}{0.4\linewidth}
		\centering
		\includegraphics[scale=.5]{example-image-a}
		\caption{说明}\label{}
	\end{subfigure}
	\hfil
	\begin{subfigure}{0.4\linewidth}
		\centering
		\includegraphics[scale=.5]{example-image-b}
		\caption{演示}\label{}
	\end{subfigure}
	\caption{图片排版示例}
\end{figure}

\subsubsection{实验装置}%四级标题

\zhlipsum[4]


\chapter{实验装置原理}%大标题

\zhlipsum[4]

\section{图片测试}

海上生明月,天涯共此时 good moning。
\zhlipsum[1]


\begin{figure}[h!]
	\centering
	\begin{subfigure}{0.4\linewidth}
		\centering
		\includegraphics[scale=.5]{example-image-a}
		\caption{说明}\label{}
	\end{subfigure}
	\hfil
	\begin{subfigure}{0.4\linewidth}
		\centering
		\includegraphics[scale=.5]{example-image-b}
		\caption{演示}\label{}
	\end{subfigure}
	\caption{图片排版示例}
\end{figure}

\section{公式测试}

劳仑衣普桑,认至将指点效则机,最你更枝。想极整月正进好志次回总般,段
然取向使张规军证回,世市总李率英茄持伴。用阶千样响领交出
\begin{equation}\label{key}
	\Delta x \geqslant \frac{p c}{E} \frac{\hbar}{m c}=\frac{v}{c} \frac{\hbar}{m c}
\end{equation}
最你更枝。想极整月正进好志次回总般,段
然取向使张规军证回,世市总李率英茄持伴。用阶千样响领交出
\begin{equation}\label{key}
	S=\int \mathcal{L}\left(\phi, \partial_{\mu} \phi\right) \mathrm{d}^{4} x
\end{equation}
最你更枝。想极整月正进好志次回总般,段
然取向使张规军证回,
\begin{equation}\label{key}
	\left[\frac{\partial \mathcal{L}}{\partial \phi}\left(\partial_{\nu} \phi\right)+\frac{\partial \mathcal{L}}{\partial\left(\partial_{\mu} \phi\right)} \partial_{\nu}\left(\partial_{\mu} \phi\right)\right] \delta x^{\nu}+\mathcal{L} \partial_{\nu}\left(\delta x^{\nu}\right)=\left(\partial_{\nu} \mathcal{L}\right) \delta x^{\nu}+\mathcal{L} \partial_{\nu}\left(\delta x^{\nu}\right)
\end{equation}
世市总李率英茄持伴。用阶千样响领交出
\begin{equation}\label{key}
	\left(\begin{array}{l}
		\phi_{1} \\
		\phi_{2}
	\end{array}\right) \longrightarrow\left(\begin{array}{l}
		\phi_{1}^{\prime} \\
		\phi_{2}^{\prime}
	\end{array}\right)=\left(\begin{array}{cc}
		\cos \theta & -\sin \theta \\
		\sin \theta & \cos \theta
	\end{array}\right)\left(\begin{array}{l}
		\phi_{1} \\
		\phi_{2}
	\end{array}\right)
\end{equation}
知易众美布会亲军千,件声坑志支较学。农六斯南何记子机量
各然,快写线信权间越部色,象照屈型部物治地长。难要技第对老共达质标压心,
\begin{equation}\label{key}
	\begin{aligned}
		\left\langle q^{\prime}\left|\mathrm{e}^{-\mathrm{i} H\left(t^{\prime}-t\right)}\right| q\right\rangle=& \int\left[\frac{\mathrm{d} p \mathrm{~d} q}{2 \pi}\right] \exp \left\{\mathrm{i} \int_{t}^{t^{\prime}} \mathrm{d} t[p \dot{q}-H(p, q)]\right\} \\
		& \equiv \lim _{n \rightarrow \infty} \int\left(\frac{\mathrm{d} p_{1}}{2 \pi}\right) \cdots\left(\frac{\mathrm{d} p_{n}}{2 \pi}\right) \int \mathrm{d} q_{1} \cdots \mathrm{d} q_{n-1} \\
		& \exp \left\{\mathrm{i} \sum_{i=1}^{n} \delta t\left[p_{i}\left(\frac{q_{i}-q_{i-1}}{\delta t}\right)-H\left(p_{i}, \frac{q_{i}+q_{i-1}}{2}\right)\right]\right\}
	\end{aligned}
\end{equation}
\zhlipsum[8]



\section{各种引用测试}

\subsection{图片引用}

\begin{figure}[h!]
	\centering
	\begin{subfigure}{0.4\linewidth}
		\centering
		\includegraphics[scale=.5]{example-image-c}
		\caption{说明}\label{图片引用测试a}
	\end{subfigure}
	\hfil
	\begin{subfigure}{0.4\linewidth}
		\centering
		\includegraphics[scale=.5]{example-image}
		\caption{演示}\label{图片引用测试b}
	\end{subfigure}
	\caption{超长长长长长长长长长长长长\\ 长长长长长长长长长长长长长长长长长长长长长长长长\\ 长长长长长长长长长长长长长长长长长长长长标题,图片引用测试}\label{图片引用测试}
\end{figure}

参见图 \ref{图片引用测试} ,参见子图 \ref{图片引用测试a},见子图 \ref{图片引用测试b}。长标题可手动换行。



\subsection{表格引用}


\begin{table}[h!]
	\centering
	\caption{表格引用测试}\label{表格引用测试}
	\begin{subtable}{0.4\linewidth}
		\centering
		\begin{tblr}{
				row{odd} = {bg=azure8},
				row{1}   = {bg=azure3, fg=white, font=\sffamily},
			}
			Alpha & Beta    & Gamma \\
			Delta & Epsilon & Zeta  \\
			Eta   & Theta   & Iota  \\
			Kappa & Lambda  & Mu    \\
			Nu Xi Omicron & Pi Rho Sigma & Tau Upsilon Phi \\
		\end{tblr}
		\caption{测试a}\label{表格引用测试a}
	\end{subtable}
	\hfill
	\begin{subtable}{0.4\linewidth}
		\centering
		$\begin{tblr}{
				column{1} = {mode=text},
				column{3} = {mode=dmath},
			}
			\hline
			Alpha   & \frac12 & \frac12 \\
			Epsilon & \frac34 & \frac34 \\
			Iota    & \frac56 & \frac56 \\
			Epsilon & \frac34 & \frac34 \\
			\hline
		\end{tblr}$
		\caption{测试b}\label{表格引用测试b}
	\end{subtable}
\end{table}


参见表 \ref{表格引用测试},参见子表\ref{表格引用测试a},见\ref{表格引用测试b}。



\subsection{注释引用}

虚室生白\footnote{注释引用测试a \label{注释引用测试a} },吉祥止止\footnote{注释引用测试b \label{注释引用测试b} }。参见注释 \ref{注释引用测试a},见注释 \ref{注释引用测试b}。

\subsection{公式引用}

\begin{equation}\label{公式引用测式}
	1+2=3
\end{equation}
公式 (\ref{公式引用测式}) 参见。 
\begin{subequations}
	\begin{align}
		1+1&>1 \label{公式引用测式a}\\
		1+2+3+4 & > 4 \label{公式引用测式b}
	\end{align}
\end{subequations}
见公式 \eqref{公式引用测式a},见 \eqref{公式引用测式b}。



\section{图表注释测试}
水厂共当而面三张,白家决空给意层般,单重总歼者新。每建马先口住月大,
究平克满现易手,省否何安苏京。
\subsection{表注释}
\begin{table}[h!]
	\centering
	\caption{表头加注释 \protect\footnotemark ,表头添加参考文献\cite{叶普1993关于对瞬心的动量矩定理}}
	\begin{tblr}{
			hline{1,Z} = {2pt},
			hline{2,Y} = {1pt},
			hline{3-X} = {dashed},
		}
		Alpha   & Beta & Gamma   & Delta   \\
		Epsilon & Zeta  & Eta     & Theta   \\
		Iota    & Kappa & Lambda  & Mu      \\
		Nu      & Xi    & Omicron & Pi      \\
		Rho     & Sigma & Tau     & Upsilon \\
		Phi     & Chi   & Psi     & Omega   \\
	\end{tblr}
\end{table}
\footnotetext{表头加注释}

\zhlipsum[2-3]


\begin{table}[h!]
	\caption{A table with notes}\label{tab:tablenotes}
	\centering
	\begin{threeparttable}
		\begin{tabular}{*4{c}}\toprule
			Table head\tnote{a} & Table head & Table head & Table head\tnote{b} \\ \midrule
			Some values & Some values & Some values & Some values \\
			Some values & Some values & Some values & Some values \\ \bottomrule
		\end{tabular}
		\begin{tablenotes}
			\footnotesize
			\item[a] The quick brown fox jumps over the lazy dog. The quick brown fox jumps over the lazy dog.
			\item[b] The quick brown fox jumps over the lazy dog.
		\end{tablenotes}
	\end{threeparttable}
\end{table}


\subsection{图片注释}

\begin{figure}[h!]
	\centering                        
	\includegraphics[width=0.3\textwidth]{example-image-duck}
	\caption[frog]{frog \footnotemark 添加参考文献\cite{2003张量分析} }
	\label{fig:pic1}
\end{figure}
\footnotetext{frog图片注释测试}



\section{理论计算}%二级标题

\zhlipsum[2]

\subsection{理论预言}%三级标题

\zhlipsum[3]

\subsubsection{结果近似}%四级标题

\zhlipsum[4]