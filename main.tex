\documentclass{ZDSS}


\addbibresource[location=local]{date/refer.bib} %参考文献源文件


\begin{document}


%基本信息填写
\BasicInfomation{
	degree_level = {master}, %学位层级:硕士(master)/ 博士(doctor)
	degree_type = {学术学位}, %学位类型:专业学位(专硕)/学术学位(学硕),博士不需要填写
	thesis_title_zh = {世界一流的外焦里嫩烤地瓜\\技术流程}, % 论文题目,中文
	thesis_title_zh_text = {世界一流的外焦里嫩烤地瓜技术流程}, % 论文题目,中文,纯文字
	thesis_title_en = {World class technology process for roasting sweet potatoes with\\ outer coke and inner tender}, % 论文题目,英文
	student_name = {李四}, % 学生姓名
	student_number = {22210802720}, %学生学号
	teacher_name = {张三, 王二},  % 导师姓名,多个导师,请用英文逗号分隔
	teacher_title = {副教授, 特聘研究员},  % 导师职称,多个导师,请用英文逗号分隔
	college = {理学院},  % 学院
	major = {物理学}, % 专业
	research_orientation = {烤地瓜}, % 研究方向
	complete_date = {2024年11月09日},  % 完成日期
	school_name = {西藏大学}, %学校名称
	school_code = {10694}, %学校代码
	security_classification = {公开}, %论文密级
	terms_create = {true}, %是否创建术语表: true/false,若选择创建,则在文中使用 \myterm[]{中文}{English} 命令可自动添加术语表条目
	print_version = {true}, %是否生成打印版本的PDF: true/false,若为真,则在生成 PDF 时会自动添加空白页,以适应双面打印的需要
	anonymously_review = {false}, %是否启用盲审格式
}



\CoverDraw %封面绘制


\AuthorStatement %作者声明


%中英文摘要


\ZhAbstract %中文摘要相关命令,勿删


%硕士学位论文应附1000字以上的中文摘要。



\zhlipsum[1-4]



\ZhAbstractKeyWord{关键词,关键词,关键词,关键词}







\EnAbstract %英文摘要相关命令,勿删



\lipsum[1-10]



\EnAbstractKeyWord{KEY WORDS, KEY WORDS, KEY WORDS, KEY WORDS}











%目录
\TOCautoCreate %目录自动创建



%正文开始
\include{1/绪论}





\RefautoCreate %参考文献自动创建


%附录相关内容
\AppendixStart %附录开始
\chapter{测试}


\zhlipsum[1]


\section{测试}


\zhlipsum[2-3]


\subsection{测试}


\zhlipsum[3-5]


\subsection{测试}


\zhlipsum[3]


\section{测试}


\zhlipsum[2]


\subsection{测试}


\zhlipsum[3]

\chapter{测试}


\zhlipsum[4]


\section{测试}


\zhlipsum[5]


\subsection{测试}


\zhlipsum[6]






\OtherStart %其它内容开始
\chapter{攻读学位期间科研成果}


\section*{论文发表}
\begin{refsection}[date/results.bib]
	%通过 \nocite{} 命令加入想要显示的文献
	%若新加的没有显示,可清理缓存后重新编译
	\nocite{Zhang:2017eah}
	\nocite{Wang:2017ukk}
	\nocite{Jusufi:2017drg}
	\printbibliography[heading=none]
\end{refsection}


\section*{会议报告}
\begin{enumerate}
	\item
	口头报告:``When Primordial Black Holes from Sound Speed Resonance Meet
	a Stochastic Background of Gravitational Waves'' , Journal Club at Department
	of Astronomy, USTC, Hefei, China, December 29, 2019;
	\item 
	海报:``Sound Speed Resonance Mechanism in Curvaton'', 2019 Annual Meeting
	of the Chinese Physical Society, Division of Gravitation and Relativistic Astrophysics, Guiyang, China, July 16, 2019;
\end{enumerate}

\section*{项目申请}
\begin{enumerate}
	\item
	西藏大学研究生“高水平人才培养计划”项目
	\item 
	西藏大学校级青年培育项目
\end{enumerate}

 %攻读学位期间发表的学术论文目录
\chapter{致 \quad 谢}


\zhlipsum[1]

\zhlipsum[7]


 %致谢




\end{document}
